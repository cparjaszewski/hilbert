%% Created by Maple 14.00 (IBM INTEL NT)
%% Source Worksheet: Thesis.mws
%% Generated: Mon Jul 04 12:07:26 2011
\documentclass{article}
\usepackage{maple2e}
 \def\emptyline{\vspace{12pt}}
\DefineParaStyle{Heading 1}
\DefineParaStyle{Heading 2}
\DefineParaStyle{Heading 2}
\DefineParaStyle{Heading 3}
\DefineParaStyle{Heading 4}
\DefineParaStyle{Heading 4}
\DefineParaStyle{Normal}
\DefineParaStyle{Normal}
\DefineCharStyle{2D Comment}
\DefineCharStyle{2D Math}
\DefineCharStyle{Plot Text}
\begin{document}
\pagestyle{empty}

\section{Numerical calculation of the Hilbert transform applied for
understanding and solving the Kramers-Kronig relations in nonlinear
optics}

\subsection{0. Abstract}

\subsubsection{Dedication}

\begin{maplegroup}
\begin{flushleft}
To my Family.
\end{flushleft}

\end{maplegroup}

\subsubsection{Summary}

\begin{maplegroup}
\begin{flushleft}
The motivation for this work comes from the real problem state by
physicists and chemists building the valid theoretical models to
describe the interaction between light and matter due to both
non-intensive and intensive light radiation. The starting point is the
fundamental physical law - the causality principle - which states that
the effect cannot precede the cause. This simple assumption leads us
the Titchmarsh theorem about the Hilbert Transform - and together with
advanced signal response-theory we can investigate the properties of
the medium response to the periodic input signal - both in time and
frequency domain. Modifications of the Hilbert Transform for the
optical susceptibility 
\mapleinline{inert}{2d}{chi(omega);}{%
$\chi (\omega )$%
} are known as the Kramers-Kronig relations - which are defined for
both real and imaginary part of susceptibility in the frequency-domain
 :
\end{flushleft}

\end{maplegroup}
\begin{maplegroup}
\begin{flushleft}
\mapleinline{inert}{2d}{Re(chi(omega)) =
2/Pi*int(Omega*Im(chi(Omega))/(Omega^2-omega^2),Omega = -infinity ..
infinity);}{%
$\Re (\chi (\omega ))=\frac {2\,\int _{ - \infty }^{\infty }
\frac {\Omega \,\Im (\chi (\Omega ))}{\Omega ^{2} - \omega ^{2}}
\,d\Omega }{\pi }$%
}
\end{flushleft}

\end{maplegroup}
\begin{maplegroup}
\begin{flushleft}
\mapleinline{inert}{2d}{Im(chi(omega)) =
2*omega/Pi*int(Re(chi(Omega))/(Omega^2-omega^2),Omega = -infinity ..
infinity);}{%
$\Im (\chi (\omega ))=\frac {2\,\omega \,\int _{ - \infty }^{
\infty }\frac {\Re (\chi (\Omega ))}{\Omega ^{2} - \omega ^{2}}\,
d\Omega }{\pi }$%
}
\end{flushleft}

\end{maplegroup}
\begin{maplegroup}
\begin{flushleft}
where the integration uses the Cauchy principal value method. In the
thesis we handle two problems. First concerns numerical problems with
calculation of such singular and improper integral. The second problem
concerns the questions stated by physicists - how to properly use this
mathematical tools in a typical experiment and model construction in
optical research. We present the comparison of several implementations
of numerical calculations of the Hilbert transform:
- Numerical trapezoidal rule mixed with the Simpson's rule and the
cubic interpolation
- Newton-Cotes qudrature of sixth degree
- Clenshaw-Curtis quadrature
- Hilbert transform based on fast Hartley transforms
- method based on approximation with the orthonormal Hermite
polynomials and Hermite functions
- method based on approximation with the Fourier series.
We also test the out-of-the-box MATLAB-implemented routines:
- quadgk()
- hilbert() - based on FFT
The given physical models for both linear and nonlinear optics are
analysized and validated. We formulate good practises for scientists
interested in subject of optical experiments. Finally we made the
conclusions about the numerical stability, advantages and
disadvantages of developed implementations in further research in
nonlinear optics.
\end{flushleft}

\end{maplegroup}
\begin{maplegroup}
\emptyline
\end{maplegroup}

\subsubsection{Keywords}

\begin{maplegroup}
\begin{flushleft}
numerical analysis, nonlinear optics, Hilbert transform,
Kramers-Kronig relations, optical dispersion relations
\end{flushleft}

\end{maplegroup}

\subsubsection{Promotors}

\subsubsection{Nonlinear optics:}

\begin{maplegroup}
\begin{flushleft}
\textbf{Professor Marek Samo�}
\end{flushleft}

\begin{flushleft}
Institute of Physical and Theoretical Chemistry
Wroclaw University of Technology, PL-50-370 Wroclaw
Wybrzeze Wyspianskiego 27, Poland
+48-71-320-4466 | E-mail: Marek.Samoc@pwr.wroc.pl
\end{flushleft}

\end{maplegroup}

\subsubsection{Numerical analysis:}

\begin{maplegroup}
\begin{flushleft}
\textbf{Pawe� Keller PhD}
\end{flushleft}

\begin{flushleft}
Group of Numerical Methods
\end{flushleft}

\begin{flushleft}
Institute of Computer Science
\end{flushleft}

\begin{flushleft}
University of Wroclaw, PL-50-383 Wroclaw
\end{flushleft}

\begin{flushleft}
ul. Joliot-Curie 15, Poland
\end{flushleft}

\begin{flushleft}
+48-71-375-7813 | E-mail: Pawel.Keller@ii.uni.wroc.pl
\end{flushleft}

\end{maplegroup}
\emptyline

\subsubsection{Author}

\begin{maplegroup}
\begin{flushleft}
\textbf{Krzysztof Parjaszewski}
\end{flushleft}

\begin{flushleft}
Institute of Computer Science
University of Wroc�aw, PL-50-383 Wroc�aw
ul. Joliot-Curie 15, Poland
+48-660-070-043 | E-mail: Krzysztof.Parjaszewski@Gmail.Com
\end{flushleft}

\end{maplegroup}

\subsubsection{Reviewer}

\begin{maplegroup}
\begin{flushleft}
Yet unknown
\end{flushleft}

\end{maplegroup}

\subsection{1. Introduction to the nonlinear optics (understanding)}

\subsubsection{1.1 What is this thesis about?}

\begin{maplegroup}
\begin{flushleft}
This thesis is a small step for better understanding the nature of
light and its interaction with matter.
\end{flushleft}

\end{maplegroup}
\begin{maplegroup}
\begin{flushleft}
The main motivation and subject of this thesis is the application of
numerical methods for better modelling and understanding the physical
models in nonlinear optics. The main experiment used during the work
on this thesis has been conducted with the Z-Scan technique on a
high-energy femtosecond laser system consisting of a Quantronix
Integra-C amplifier operating at 800 nm and a Quantronix-Palitra-FS
BIBO crystal-based optical parametric amplifier which allows us to
modify the output wavelength in range of \symbol{126}650nm up to
\symbol{126}2000nm. The laser pulses were of \symbol{126}130 fs length
and 1kHz repetition rate. Experiments were performed at the Wroclaw
University of Technology at the Institute of Physical and Theoretical
Chemistry. Numerical calculations were performed on a simple personal
laptop within MATLAB and Maple environment.
\end{flushleft}

\end{maplegroup}
\begin{maplegroup}
\begin{flushleft}
The main numerical results concern the usage of the the Kramers-Kronig
relations and the discussion about its application for nonlinear
optics. Firstly the simple linear Kramers-Kronig relation has been
derived and reviewed, then is has been developed into simple nonlinear
case. Next the perturbative approach and other modifications in
discussed model are reviewed. In each of these examples, an attention
is paid to the numerical issues. The constructed numerical
calculations are compared and discussed and further good practices for
physicists and chemists are formulated.
\end{flushleft}

\end{maplegroup}
\begin{maplegroup}
\begin{flushleft}
The subject of this thesis is not accidential - in last twenty years
the global interest on light application in modern devices - including
the construction of super-fast all-optical computers - increases. In
Poland - the nanophotonics research is developing in several
scientific centers and Wroclaw is one of the leading ones.
\end{flushleft}

\end{maplegroup}
\begin{maplegroup}
\begin{mapleinput}
\mapleinline{active}{1d}{To review both with Prof. MS and Dr PK.}{%
}
\end{mapleinput}

\end{maplegroup}

\subsubsection{1.2 What is the nonlinear optics?}

\begin{maplegroup}
\begin{flushleft}
In mathematics a linear function f satisfies both of the following
properties:
\end{flushleft}

\end{maplegroup}
\begin{maplegroup}
\begin{flushleft}
(1.2.1a) + additivity: : " 
\mapleinline{inert}{2d}{x,y;}{%
$x, \,y$%
} : 
\mapleinline{inert}{2d}{f(x+y) = f(x)+f(y);}{%
$\mathrm{f}(x + y)=\mathrm{f}(x) + \mathrm{f}(y)$%
},
(1.2.1b) + homogeneity:" 
\mapleinline{inert}{2d}{alpha;}{%
$\alpha $%
} : 
\mapleinline{inert}{2d}{f(alpha*x) = alpha*f(x);}{%
$\mathrm{f}(\alpha \,x)=\alpha \,\mathrm{f}(x)$%
}.
\end{flushleft}

\end{maplegroup}
\begin{maplegroup}
\begin{flushleft}
Sometimes another name for a function is used: "system". The nonlinear
system is so a function which does not satisfies both additivity and
homogenity simultaneosly. Another definition says that nonlinear
system does not follow the superposition rule. The main area of
interest for nonlinear optics is the behaviour of optical wave in a
nonlinear medium. In such a case the superposition principle no longer
holds, and therefore the response of a system/function depends
nonlinearly on the input signal. In practice the nonlinearity shows
only with very high intensities of light, so this theory is only
applied to high intensity devices - such as modern lasers or optical
fibers. Nonlinear optics is strongly related to physics, biology and
chemistry but also mathematics and computer science, because modelling
the light behaviour is an open problem still being solved
interdisciplinary - what has been showed on the Picture
1.2.1:\textit{}
\end{flushleft}

\end{maplegroup}
\begin{maplegroup}
\begin{flushleft}
\end{flushleft}

\begin{flushleft}
\QTR{Plot Text}{Picture 1.2.1 - Nonlinear optics derives from many foundamentary
disciplines: biology, chemistry, math, physics and computer science.}
\end{flushleft}

\end{maplegroup}
\begin{maplegroup}
\begin{flushleft}
An comprehensive, well prepared introduction to the nonlinear optics
can be found in the \textit{Nonlinear Optics} written by Robert Boyd
[2].
\end{flushleft}

\end{maplegroup}

\subsubsection{1.3 What are the most important nonlinear optical
properties and phenomena?}

\begin{maplegroup}
\begin{flushleft}
Experiments, which investigates the nonlinear nature of light are
similar to those in classical optics, but they differ in the intensity
of source radiation. Under the influence of the powerful and ultrafast
lasers all optical properties and phenomena such as refraction,
diffraction, interference, dyspersion, scattering etc. tend to modify.
With pulses lasting from 10 femtoseconds up to 10 picoseconds and
energy per pulse between 1 mJ and 1000 mJ laser produces a single
pulse with a power of giga-, tera- or even perawatts. Influenced by a
such huge power of the light beam - matter shows its nonlinear
properties. 
\end{flushleft}

\end{maplegroup}
\begin{maplegroup}
\begin{flushleft}
We define the following classification of the linear and nonlinear
optical phenomena:
- linear phenomena
- second order nonlinear phenomena
\end{flushleft}

\begin{flushleft}
- third order nonlinear phenomena
\end{flushleft}

\begin{flushleft}
- nonlinear phenomena of higher orders
\end{flushleft}

\end{maplegroup}
\begin{maplegroup}
\begin{flushleft}
\end{flushleft}

\begin{flushleft}
\QTR{Plot Text}{Picture 1.3.1 - schematic diagram of nonlinear optical phenomena -
the input wavelength differs from the output signal, which occurs only
for strong optical signals.}
\end{flushleft}

\end{maplegroup}
\begin{maplegroup}
\begin{flushleft}
The basis of such classification is the equation defined for the
electric polarization by the classical electromagnetism. The
polarization density vector 
\mapleinline{inert}{2d}{[C/(m^2)];}{%
$[\frac {C}{m^{2}}]$%
} is approximated by Taylor series: 
\end{flushleft}

\end{maplegroup}
\begin{maplegroup}
\begin{flushleft}
(1.3.1) 
\mapleinline{inert}{2d}{P[i]/epsilon[0] = P[i](0)/epsilon[0]+sum(chi[ij]^[1]*E[j],j = 1 ..
3)+sum(sum(chi[ijk]^[2]*E[j]*E[k],j = 1 .. 3),k = 1 ..
3)+sum(sum(sum(chi[ijkl]^[3]*E[j]*E[k]*E[l],j = 1 .. 3),k = 1 .. 3),l
= 1 .. 3);}{%
$\frac {{P_{i}}}{{\varepsilon _{0}}}=\frac {{P_{i}}(0)}{{
\varepsilon _{0}}} + (\sum _{j=1}^{3}\,{\chi _{\mathit{ij}}}^{[1]
}\,{E_{j}}) + (\sum _{k=1}^{3}\,(\sum _{j=1}^{3}\,{\chi _{
\mathit{ijk}}}^{[2]}\,{E_{j}}\,{E_{k}})) +  \left(  \! \sum _{l=1
}^{3}\,(\sum _{k=1}^{3}\,(\sum _{j=1}^{3}\,{\chi _{\mathit{ijkl}}
}^{[3]}\,{E_{j}}\,{E_{k}}\,{E_{l}})) \!  \right) $%
} + ...    for 
\mapleinline{inert}{2d}{i = 1,2,3;}{%
$i=1, \,2, \,3$%
} with omission of 
\mapleinline{inert}{2d}{1/n!;}{%
$\frac {1}{n\mathrm{!}}$%
} factors in the power expansion,
\end{flushleft}

\end{maplegroup}
\begin{maplegroup}
\begin{flushleft}
where:
\end{flushleft}

\begin{flushleft}
\mapleinline{inert}{2d}{chi^[1];}{%
$\chi ^{[1]}$%
} : 
\mapleinline{inert}{2d}{[m^2/(V^2)];}{%
$[\frac {m^{2}}{V^{2}}]$%
} - the linear susceptibility tensor;
\end{flushleft}

\begin{flushleft}
\mapleinline{inert}{2d}{chi^[2];}{%
$\chi ^{[2]}$%
} : 
\mapleinline{inert}{2d}{[m^2/(V^2)];}{%
$[\frac {m^{2}}{V^{2}}]$%
} - the second order nonlinear susceptibility tensor - responsible for
phenomena such as electro-optic rectification (EOR), second-harmonic
generation (SHG), generation of sum and differential frequencies (SFG
and DFG) and Pockels electro-optic effect;
\end{flushleft}

\begin{flushleft}
\mapleinline{inert}{2d}{chi^[3];}{%
$\chi ^{[3]}$%
} : 
\mapleinline{inert}{2d}{[m^2/(V^2)];}{%
$[\frac {m^{2}}{V^{2}}]$%
} - the third order nonlinear susceptibility tensor - responsible for
phehonema such as intensity dependent refractive index (IDRI), third
harmonic generation (THG), stimulated brillouin scattering (SBS),
stimulated raman scattering (SRS), degenerate four waves mixing (DFWM)
and nonlinear absorption [3]. 
\end{flushleft}

\end{maplegroup}
\begin{maplegroup}
\begin{flushleft}
The intensity dependent refractive index (IDRI) results from the
resonant response of an atomic system and can be described by the
equation:
\end{flushleft}

\end{maplegroup}
\begin{maplegroup}
\begin{flushleft}
(1.3.2) 
\mapleinline{inert}{2d}{n = n[0]+[n[2]]*`<,>`(E^2);}{%
$n={n_{0}} + [{n_{2}}]\,\langle E^{2}\rangle $%
},
\end{flushleft}

\end{maplegroup}
\begin{maplegroup}
\begin{flushleft}
where: 
\end{flushleft}

\begin{flushleft}
\mapleinline{inert}{2d}{n[0];}{%
${n_{0}}$%
} - the weak-field refractive index;
\end{flushleft}

\begin{flushleft}
\mapleinline{inert}{2d}{[n[2]];}{%
$[{n_{2}}]$%
} - the second order index of refraction;
\end{flushleft}

\begin{flushleft}
\mapleinline{inert}{2d}{`<,>`(E^2);}{%
$\langle E^{2}\rangle $%
} - means the time average around E.
\end{flushleft}

\end{maplegroup}
\begin{maplegroup}
\begin{flushleft}
\mapleinline{inert}{2d}{[n[2]];}{%
$[{n_{2}}]$%
} is a new optical constant which describes the increasement of the
refractive index in the relation to the intensity of optical signal. 
\end{flushleft}

\end{maplegroup}
\begin{maplegroup}
\begin{flushleft}
The arisement of the refractive index is sometimes called the optical
Kerr effect. It is a similar effect to the Kerr electro-optic effect,
which is a modification of the refraction index under the influence of
the strong electric field applied to the medium. In the optical case
the change of refraction index comes only from the influence of very
intensive light beam, typically from the laser system. The theory
beneath this effect has its roots in some quantum mechanics effects
(e.g. the dynamic Burstein-Moss effect) and as the nanophotonics is
relatively new scientific domain, there are only more or less
confirmed hypothesis being under investigation.
\end{flushleft}

\end{maplegroup}
\begin{maplegroup}
\begin{flushleft}
Several effects come as the result from the indensity dependent
refractive index:
a) self-focusing of light - focusing the light beam within the
nonlinear medium
\end{flushleft}

\begin{flushleft}
b) two beam coupling - is the energy transfer between two light beams
interaction within the nonlinear medium
\end{flushleft}

\begin{flushleft}
c) optical phase conjugation - is responsible for reversing the
typical abberation effects 
d) all-optical switching - is the very interesting effect which allows
to control the flow of one beam by another. In other words it is a
control of light by light.  
\end{flushleft}

\end{maplegroup}
\begin{maplegroup}
\begin{flushleft}
The nonlinear absorption may concerns the absorption of energy from
two or more photons influencing matter in the very same moment of time
or the absorption of light under the heavy light intensity. Both
linear and nonlinear absorption process has the property of spatial
selectivity, which means that the with different angle of light beam
incidenting on matter we get different absorption coefficients - which
is related to the tensor nature of the optical susceptibility. Also
the polarization geometry of the light beam plays important role here
[5].
\end{flushleft}

\end{maplegroup}
\begin{maplegroup}
\begin{flushleft}
\textit{Description of determination the nonlinear optical properties
and describing the NLO phenomena in several sentenses.}
\end{flushleft}

\end{maplegroup}

\subsubsection{1.4 Why is nonlinear optics so important?}

\begin{maplegroup}
\begin{flushleft}
Nonlinear optics is the widely investigated area in nanophotonics,
because a material with a high nonlinear refractive index will be a
key to to build an all-optical-transistor and such device should work
as an all-optical logic gate. That will lead us to create the
ultra-fast optical devices, for instance the fotonic computer - where
instead of electrons, fotons will be used to perform computations with
a very high speed. Some scientists suggest that the taming of
all-optical-switching phenomenom is the Holy Grail of modern photonics
[3].
\end{flushleft}

\end{maplegroup}
\begin{maplegroup}
\begin{flushleft}
On the other hand - the nonlinear absorption, which is strongly
related to the nonlinear refraction index - is an important obstacle
to build such a fast all-optical-device. Despite ot that, nonlinear
absorption still may be very useful in nanofabrication, biophotonics,
microscopy, information storage and other technologies using short
laser pulses[4]. 
\end{flushleft}

\end{maplegroup}
\begin{maplegroup}
\begin{flushleft}
The interest in nonlinear optics comes from both the basic research
and industry. The basic research and fundamental questions in material
research concerns the nature of light, optical properties of
molecules, the strength and properties of chemical bonds and theory to
determine both linear and nonlinear optical properties insead of
performing myriads of experiments. The industrial research interests
in nonlinear optics are mostly focused in the area of developing new
optical devices (detectors, engines, logic gates, fibers, lasers and
more) and the usage of intensive light beams for various types of
screening and determing the 3D information about the geometrical of
the investigated material.
\end{flushleft}

\end{maplegroup}
\begin{maplegroup}
\begin{flushleft}
\textit{Description of} \textit{the most expected results of
understanding the nonlinear optics.}
\end{flushleft}

\end{maplegroup}

\subsubsection{1.5 Why should we use Kramers Kronig relations here?}

\begin{maplegroup}
\begin{flushleft}
Kramers Kronig relations are the fundamental tools in the exploration
of the material optical properties and together with other classical
physics and quantum mechanics predictions can be used to construct and
validate theoretical models. They are originated from the most basic
but also very important law in physics - the causality principle. From
the philosophical perspective it states that each effect must be
preceded by a cause. In computer science this principle is somehow
paraphrased in a discussion of formal deterministic and
nondeterministic automatas. We believe that only deterministic
automatas really exist. 
\end{flushleft}

\end{maplegroup}
\begin{maplegroup}
\begin{flushleft}
(1.5.1a) 
\mapleinline{inert}{2d}{Psi(t);}{%
$\Psi (t)$%
} = 
\mapleinline{inert}{2d}{IFT(Psi(omega));}{%
$\mathrm{IFT}(\Psi (\omega ))$%
} 
(1.5.1b) " t \TEXTsymbol{<} 0 : 
\mapleinline{inert}{2d}{Psi(t) = 0;}{%
$\Psi (t)=0$%
} 
\end{flushleft}

\end{maplegroup}
\begin{maplegroup}
\begin{flushleft}
The causability principle looks obvious, but in the field of nonlinear
optics - combined together with the application of the inverse Fourier
transform of any nonlinear quantity spectrum: 
\mapleinline{inert}{2d}{Psi(omega);}{%
$\Psi (\omega )$%
}, \textbf{R} ' 
\mapleinline{inert}{2d}{omega;}{%
$\omega $%
}  - it states that the time-domain signal 
\mapleinline{inert}{2d}{Psi(t);}{%
$\Psi (t)$%
} calculated with (1.5.1a) from frequency-domain must equal zero for
any negative time argument. This statement can be used in validation
of any spectrum 
\mapleinline{inert}{2d}{Psi(omega);}{%
$\Psi (\omega )$%
}, \textbf{R} ' 
\mapleinline{inert}{2d}{omega;}{%
$\omega $%
} - what is important when we only measure the physical quantity in a
specific range \$ 
\mapleinline{inert}{2d}{a,b:;}{%
$a, \,b$%
} : 
\mapleinline{inert}{2d}{a < omega;}{%
$a < \omega $%
} and 
\mapleinline{inert}{2d}{omega < b;}{%
$\omega  < b$%
}. We simply cannot measure the whole real spectrum \textbf{R} ' 
\mapleinline{inert}{2d}{omega;}{%
$\omega $%
} and in this case we need to perform the validation of approximated
or assumed spectrum 
\mapleinline{inert}{2d}{Psi(omega);}{%
$\Psi (\omega )$%
}.
\end{flushleft}

\end{maplegroup}
\begin{maplegroup}
\begin{flushleft}
Kramers Kronig relations - which will be derived in the following
chapter - connects the real and imaginary part of the Fourier
transform of valid response signal. It is said that one can derive the
real part of the determined spectrum 
\mapleinline{inert}{2d}{Psi(omega);}{%
$\Psi (\omega )$%
}. What is even more important - this relation is bidirectional - so
with the Kramers Kronig relations one can derive the imaginary part
from real. This hypothesis has been formulated in 1948 by Edward
Charles Titchmarsh - with the tacit assumption, that both 
\mapleinline{inert}{2d}{Psi(t);}{%
$\Psi (t)$%
} and 
\mapleinline{inert}{2d}{Psi(omega);}{%
$\Psi (\omega )$%
} are the square-integrable functions and more - the 
\mapleinline{inert}{2d}{Psi(omega);}{%
$\Psi (\omega )$%
} holomorphic in the upper-plane. This simple theorem - is widely
investigated area in both the nonlinear and linear optics - to connect
the refraction index togheter with absorption coefficient, which are
th real and imaginary parts of the optical susceptibility
respectively.
\end{flushleft}

\end{maplegroup}
\begin{maplegroup}
\begin{flushleft}
\textit{Just few words about the hopes of using K-K relations also in
nonlinear optics.}
\end{flushleft}

\end{maplegroup}

\subsubsection{Notes}

\begin{maplegroup}
\begin{flushleft}
1. "Znajomo�� pe�nej funkcji dielektrycznej 
\mapleinline{inert}{2d}{epsilon(omega) = epsilon[1](omega)+i*epsilon[2](omega);}{%
$\varepsilon (\omega )={\varepsilon _{1}}(\omega ) + i\,{
\varepsilon _{2}}(\omega )$%
} daje nam klasycznie pe�ny opis zjawisk towarzysz�cych rozchodzeniu
si� fali elektromagnetycznej w materii."
2. Teori� fitowania lorentzian�w opisuje regresja nieliniowa.
\end{flushleft}

\end{maplegroup}
\begin{maplegroup}
\begin{flushleft}
\textit{Some notes in both polish and english.}
\end{flushleft}

\end{maplegroup}

\subsection{2. Problem definition (understanding)}

\subsubsection{2.1 Linear model}

\begin{maplegroup}
\begin{flushleft}
Basically in the linear optical model we assume, that the response of
a medium depends linearly on the input. The similar presumption is
made in many fields of physics - for instance in magnetostatics we
linearly connect the current density \textbf{J} with the electric
field \textbf{E} vector multiplied by a conductivity parameter 
\mapleinline{inert}{2d}{sigma;}{%
$\sigma $%
} - which is called the Kirchhoff's reformulation of the Ohm's Law:
\end{flushleft}

\end{maplegroup}
\begin{maplegroup}
\begin{flushleft}
(2.1.1) \textbf{J} = 
\mapleinline{inert}{2d}{sigma;}{%
$\sigma $%
} \textbf{E}
\end{flushleft}

\end{maplegroup}
\begin{maplegroup}
\begin{flushleft}
As it will be shown later, such relations hold for sufficiently small
intensities - but the definition of 'sufficiently small' will not be
further specified. In the time-domain it is stated that an external
sinusoidal electromagnetic radiation of a light beam - \textbf{the
input} - is defined by a field vector:
\end{flushleft}

\end{maplegroup}
\begin{maplegroup}
\begin{flushleft}
(2.1.2) \textbf{E}(\textbf{r}, t) = 
\mapleinline{inert}{2d}{1/r;}{%
$\frac {1}{r}$%
} 
\mapleinline{inert}{2d}{E[0];}{%
${E_{0}}$%
}cos(\textbf{k r - 
\mapleinline{inert}{2d}{omega;}{%
$\omega $%
}} t + 
\mapleinline{inert}{2d}{phi;}{%
$\phi $%
}), 
\end{flushleft}

\end{maplegroup}
\begin{maplegroup}
\begin{flushleft}
where:
\textit{\textbf{{\small k - wave vector (radians per meter)
r - position vector (meters) 
t - time (seconds)}}}
\end{flushleft}

\begin{flushleft}
\mapleinline{inert}{2d}{omega;}{%
$\omega $%
} - \textit{{\small angular frequency (radians per second)}}
\end{flushleft}

\begin{flushleft}
\mapleinline{inert}{2d}{phi;}{%
$\phi $%
} - \textit{{\small phase shift (radians). }}
\end{flushleft}

\end{maplegroup}
\begin{maplegroup}
\begin{flushleft}
\textbf{The output} - reponse in the form of polarization vector can
be described using the linear response theory - which is the
simplification of the Volterra series that takes only the leading
order term can be described as the following convolution:
\end{flushleft}

\end{maplegroup}
\begin{maplegroup}
\begin{flushleft}
(2.1.3) 
\mapleinline{inert}{2d}{P[1,i];}{%
${P_{1, \,i}}$%
}(\textbf{r}, t) = 
\mapleinline{inert}{2d}{epsilon[0]*sum(int(int(G[1,i,j](p-r,s-t)*E[1,j](p,s),p),s = -infinity
.. infinity),j = 1 .. 3);}{%
${\varepsilon _{0}}\, \left(  \! \sum _{j=1}^{3}\,\int _{ - 
\infty }^{\infty }\int {G_{1, \,i, \,j}}(p - r, \,s - t)\,{E_{1, 
\,j}}(p, \,s)\,dp\,ds \!  \right) $%
}  with the inner integral made through the Euclidean space 
\mapleinline{inert}{2d}{R^3;}{%
$R^{3}$%
}.[6]
\end{flushleft}

\end{maplegroup}
\begin{maplegroup}
\begin{flushleft}
where:
1 - the '1' index states for linear property of any investigated value
\mapleinline{inert}{2d}{P[1,i];}{%
${P_{1, \,i}}$%
} - 3x1 real linear polarization tensor (vector) in a
spacetime-domain, i stays for chosen spatial coordinate (x, y or z) [
\mapleinline{inert}{2d}{C/(m^2);}{%
$\frac {C}{m^{2}}$%
}]
\mapleinline{inert}{2d}{G[1,i,j];}{%
${G_{1, \,i, \,j}}$%
} - 3x3 real susceptibility tensor (matrix) defined by a linear Green
function in a spacetime-domain [1]
\mapleinline{inert}{2d}{E[1,j];}{%
${E_{1, \,j}}$%
} - 3x1 real electric field tensor (vector) in a spacetime-domain [
\mapleinline{inert}{2d}{V/m;}{%
$\frac {V}{m}$%
}]
\end{flushleft}

\begin{flushleft}
\mapleinline{inert}{2d}{epsilon[0];}{%
${\varepsilon _{0}}$%
} - electric permittivity of free space  \symbol{126}8.854187817620 � 
\mapleinline{inert}{2d}{10^(-12);}{%
$10^{( - 12)}$%
}  [
\mapleinline{inert}{2d}{C/(V*m);}{%
$\frac {C}{V\,m}$%
}]
\textit{\textbf{{\small r - position vector (meters) 
t - time (seconds)}}}
\end{flushleft}

\end{maplegroup}
\begin{maplegroup}
\begin{flushleft}
Even such linear equation is only the simplification, because we do
not assume the local field corrections [7]. It seems to be an
reasonable assumption as long as we emphasize the macroscopic
character of polarization. Only in case of constructing very small
devices or synthesizing small particles - we shall take the local
field interaction into account. Investigation of the properties of the
susceptibility tensor both in time-domain and frequency-domain will be
the main part of considerations in this thesis - it is also widely
discussed topic in the literature [2,8,9,10,11]
\end{flushleft}

\end{maplegroup}
\begin{maplegroup}
\begin{flushleft}
The Fourier transform allows us to change the polarization equation
(2.1.3) from time-domain to frequency-domain:
\end{flushleft}

\end{maplegroup}
\begin{maplegroup}
\begin{flushleft}
(2.1.4) 
\mapleinline{inert}{2d}{Phi;}{%
$\Phi $%
}\textbf{\{F}[
\mapleinline{inert}{2d}{P[1,i];}{%
${P_{1, \,i}}$%
}(\textbf{r}, t)]\} = 
\mapleinline{inert}{2d}{P[1,i];}{%
${P_{1, \,i}}$%
}(\textbf{k}, 
\mapleinline{inert}{2d}{omega;}{%
$\omega $%
})
\end{flushleft}

\end{maplegroup}
\begin{maplegroup}
\begin{flushleft}
where:
\mapleinline{inert}{2d}{Phi;}{%
$\Phi $%
} - Fourier transform operation in 3D space domain
\textbf{F} - Fourier transform operation in time domain
\mapleinline{inert}{2d}{P[1,i];}{%
${P_{1, \,i}}$%
}(\textbf{k}, 
\mapleinline{inert}{2d}{omega;}{%
$\omega $%
})\textbf{ - }3x1 complex linear polarization tensor (vector) in a
wavefrequency-domain, i stays for chosen spatial coordinate (x, y or
z) [
\mapleinline{inert}{2d}{C/(m^2);}{%
$\frac {C}{m^{2}}$%
}]\textbf{
k} - 3x1 wave tensor (vector) of the electromagnetic field [
\mapleinline{inert}{2d}{1/m;}{%
$\frac {1}{m}$%
}]
\mapleinline{inert}{2d}{omega;}{%
$\omega $%
} - angular frequency of the electromagnetic field [
\mapleinline{inert}{2d}{rad/s;}{%
$\frac {\mathit{rad}}{s}$%
}],
\end{flushleft}

\end{maplegroup}
\begin{maplegroup}
\begin{flushleft}
Due the convolution theorem [12] the Fourier transform applied to the
convolution of two given functions equals the simple multiplication of
two Fourier transforms:
(2.1.5) 
\mapleinline{inert}{2d}{Phi;}{%
$\Phi $%
}\textbf{\{F}[
\mapleinline{inert}{2d}{epsilon[0]*sum(int(int(G[1,i,j](p-r,s-t)*E[1,j](p,s),p),s = -infinity
.. infinity),j = 1 .. 3);}{%
${\varepsilon _{0}}\, \left(  \! \sum _{j=1}^{3}\,\int _{ - 
\infty }^{\infty }\int {G_{1, \,i, \,j}}(p - r, \,s - t)\,{E_{1, 
\,j}}(p, \,s)\,dp\,ds \!  \right) $%
}]\} =
\mapleinline{inert}{2d}{epsilon[0]*sum(chi[1,i,j](k,omega)*E[1,j](k,omega),j = 1 .. 3);}{%
${\varepsilon _{0}}\,(\sum _{j=1}^{3}\,{\chi _{1, \,i, \,j}}(k, 
\,\omega )\,{E_{1, \,j}}(k, \,\omega ))$%
}
\end{flushleft}

\end{maplegroup}
\begin{maplegroup}
\begin{flushleft}
where:
\mapleinline{inert}{2d}{Phi;}{%
$\Phi $%
} - Fourier transform operation in 3D space domain
\textbf{F} - Fourier transform operation in time domain
\mapleinline{inert}{2d}{chi[1,i,j];}{%
${\chi _{1, \,i, \,j}}$%
}(\textbf{k}, 
\mapleinline{inert}{2d}{omega;}{%
$\omega $%
}) - 3x3 complex susceptibility tensor (matrix) in a
wavefrequency-domain [1] 
\mapleinline{inert}{2d}{E[1,j];}{%
${E_{1, \,j}}$%
}(\textbf{k}, 
\mapleinline{inert}{2d}{omega;}{%
$\omega $%
}) - 3x1 complex electric field tensor (vector) in a
wavefrequency-domain [
\mapleinline{inert}{2d}{V/m;}{%
$\frac {V}{m}$%
}]
\end{flushleft}

\end{maplegroup}
\begin{maplegroup}
\begin{flushleft}
From (2.1.4) and (2.1.5) we get:
\end{flushleft}

\end{maplegroup}
\begin{maplegroup}
\begin{flushleft}
(2.1.6) 
\mapleinline{inert}{2d}{P[1,i];}{%
${P_{1, \,i}}$%
}(\textbf{k}, 
\mapleinline{inert}{2d}{omega;}{%
$\omega $%
}) = 
\mapleinline{inert}{2d}{epsilon[0]*sum(chi[1,i,j](k,omega)*E[1,j](k,omega),j = 1 .. 3);}{%
${\varepsilon _{0}}\,(\sum _{j=1}^{3}\,{\chi _{1, \,i, \,j}}(k, 
\,\omega )\,{E_{1, \,j}}(k, \,\omega ))$%
}
\end{flushleft}

\end{maplegroup}
\begin{maplegroup}
\begin{flushleft}
which seems to be a simpler equation - so from now we will use the
wavefrequency-domain. We will no be getting further into the physical
details of the optical susceptibility theory. For our interests it is
worth stressing, that the 
\mapleinline{inert}{2d}{G[1,i,j];}{%
${G_{1, \,i, \,j}}$%
} - 3x3 real susceptibility tensor (matrix) defined by a linear Green
function in a spacetime-domain follows the following rules:
\end{flushleft}

\end{maplegroup}
\begin{maplegroup}
\begin{flushleft}
(2.1.7a) + the causality principle " 
\mapleinline{inert}{2d}{t < 0;}{%
$t < 0$%
} : 
\mapleinline{inert}{2d}{G[1,i,j];}{%
${G_{1, \,i, \,j}}$%
}(\textbf{r}, t) = 0
(2.1.8a) + the convservation of energy law: " \textbf{r }:
\mapleinline{inert}{2d}{int(G[1,i,j](r,t),t = -infinity .. infinity) < infinity;}{%
$\int _{ - \infty }^{\infty }{G_{1, \,i, \,j}}(r, \,t)\,dt < 
\infty $%
}
\end{flushleft}

\end{maplegroup}
\begin{maplegroup}
\begin{flushleft}
The Fourier transform is defined for integrable functions and after
[13,14] we can assume the asymptotic behaviour of the susceptibility
tensor in wavefrequency-domain:
(2.1.9) F[
\mapleinline{inert}{2d}{G[1,i,j];}{%
${G_{1, \,i, \,j}}$%
}(\textbf{r} ,t)] = 
\mapleinline{inert}{2d}{chi[1,i,j];}{%
${\chi _{1, \,i, \,j}}$%
}(\textbf{k}, 
\mapleinline{inert}{2d}{omega;}{%
$\omega $%
}) = 
\mapleinline{inert}{2d}{-C/(omega^2);}{%
$ - \frac {C}{\omega ^{2}}$%
} + o(
\mapleinline{inert}{2d}{omega^(-2);}{%
$\omega ^{( - 2)}$%
})
\end{flushleft}

\end{maplegroup}
\begin{maplegroup}
\begin{flushleft}
From the physical point of view such a vanishment is true - because
for an input oscillation with a frequency much higher than any
resonant frequency - the system will have not enough time to responde
before the input signal has switched it direction, so for large 
\mapleinline{inert}{2d}{omega;}{%
$\omega $%
} the susceptibility 
\mapleinline{inert}{2d}{chi[1,i,j];}{%
${\chi _{1, \,i, \,j}}$%
}(\textbf{k}, 
\mapleinline{inert}{2d}{omega;}{%
$\omega $%
}) vanishes.
\end{flushleft}

\end{maplegroup}
\begin{maplegroup}
\begin{flushleft}
where:
C - is a material specific constant, which does not depend on 
\mapleinline{inert}{2d}{omega;}{%
$\omega $%
}.
o(
\mapleinline{inert}{2d}{omega^(-2);}{%
$\omega ^{( - 2)}$%
}) - indicates a term with asymptotic descrease stricly faster then 
\mapleinline{inert}{2d}{1/(omega^2);}{%
$\frac {1}{\omega ^{2}}$%
}.
\end{flushleft}

\end{maplegroup}
\begin{maplegroup}
\begin{flushleft}
Also from the conservation of energy law we assume, that:
(2.1.10) \$ \textbf{K \TEXTsymbol{<} 
\mapleinline{inert}{2d}{infinity;}{%
$\infty $%
} : " } \textbf{d}, 
\mapleinline{inert}{2d}{omega;}{%
$\omega $%
}: 
\mapleinline{inert}{2d}{chi[1,i,j];}{%
${\chi _{1, \,i, \,j}}$%
}(\textbf{k}, 
\mapleinline{inert}{2d}{omega;}{%
$\omega $%
}) \TEXTsymbol{<} \textbf{K}
\end{flushleft}

\end{maplegroup}
\begin{maplegroup}
\begin{flushleft}
Together with the knowledge of asymptotic behaviour from (2.1.9) we
will concluded, that the susceptibility tensor is also square
integrable
\end{flushleft}

\end{maplegroup}
\begin{maplegroup}
\begin{flushleft}
(2.1.11) 
\mapleinline{inert}{2d}{`in`(chi[1,i,j](k,omega),L) and 2.1.10 implies
`in`(chi[1,i,j](k,omega),L^2);}{%
${\chi _{1, \,i, \,j}}(k, \,\omega )\,\ \textbf{in}\ \,L
\ \textbf{and}\ \mbox{2.1}\mathrm{\ .\ }10\ \textbf{implies}\ {
\chi _{1, \,i, \,j}}(k, \,\omega )\,\ \textbf{in}\ \,L^{2}$%
} 
\end{flushleft}

\end{maplegroup}
\begin{maplegroup}
\begin{flushleft}
\textit{Description of linear model of dispersion}
\end{flushleft}

\end{maplegroup}

\subsubsection{2.2 Derivation of the Kramers-Kronig for linear model}

\begin{maplegroup}
\begin{flushleft}
Before we will focus on the final form of the Kramers-Kronig relations
- we shall stress that they are the reformulation of the Titchmarsh
theorem in area of Fourier analysis. This theorem relates real and
imaginary parts of the functions from the upper half-plane of the
Hardy space: 
\mapleinline{inert}{2d}{H^p;}{%
$H^{p}$%
} with the Hilbert transform of a function from 
\mapleinline{inert}{2d}{L^p;}{%
$L^{p}$%
}.  
\end{flushleft}

\end{maplegroup}

\subsubsection{Titchmarsh theorem:}

\begin{maplegroup}
\begin{flushleft}
\textbf{Let:
}(2.2.1) 
\mapleinline{inert}{2d}{`in`(a(t),L^2);}{%
$\mathrm{a}(t)\,\ \textbf{in}\ \,L^{2}$%
}(\textbf{R}) and "  t \TEXTsymbol{<} 0 : a(t) = 0
\textbf{Then: }
\end{flushleft}

\begin{Heading 4}
(2.2.2)\textbf{ }b(
\mapleinline{inert}{2d}{omega;}{%
$\omega $%
}) = F[a(t)] and 
\mapleinline{inert}{2d}{`in`(b,H^2);}{%
$b\,\ \textbf{in}\ \,H^{2}$%
}(\textbf{U}) 
\end{Heading 4}

\end{maplegroup}
\begin{maplegroup}
\begin{flushleft}
where: \textbf{
\mapleinline{inert}{2d}{H^2;}{%
$H^{2}$%
}} - Hardy space of the holomorphic functions with a norm: sup
\{y\TEXTsymbol{>}0\} \textbf{ 
\mapleinline{inert}{2d}{[int(abs(f(x+i*y))^2,x)]^(1/2);}{%
$[\int  \left|  \! \,\mathrm{f}(x + i\,y)\, \!  \right| ^{2}\,dx]
^{(\frac {1}{2})}$%
}}\TEXTsymbol{<}\textbf{ 
\mapleinline{inert}{2d}{infinity;}{%
$\infty $%
}
U} - the upper-half complex plane, 
F - Fourier transform functional
\end{flushleft}

\end{maplegroup}
\begin{maplegroup}
\begin{flushleft}
(2.2.3a) 
\mapleinline{inert}{2d}{Re(b(omega)) = 1/pi*int(Im(b(Omega))/(Omega-omega),Omega = -infinity
.. infinity);}{%
$\Re (\mathrm{b}(\omega ))=\frac {1\,\int _{ - \infty }^{\infty }
\frac {\Im (\mathrm{b}(\Omega ))}{\Omega  - \omega }\,d\Omega }{
\pi }$%
}
(2.2.3.b) 
\mapleinline{inert}{2d}{Im(b(omega)) = 1/(-pi)*int(Re(b(Omega))/(Omega-omega),Omega =
-infinity .. infinity);}{%
$\Im (\mathrm{b}(\omega ))=\frac {1\,\int _{ - \infty }^{\infty }
\frac {\Re (\mathrm{b}(\Omega ))}{\Omega  - \omega }\,d\Omega }{
 - \pi }$%
}
\end{flushleft}

\end{maplegroup}
\begin{maplegroup}
\begin{flushleft}
where:
\mapleinline{inert}{2d}{Re(b(omega));}{%
$\Re (\mathrm{b}(\omega ))$%
} - real part of b(
\mapleinline{inert}{2d}{omega;}{%
$\omega $%
})
\end{flushleft}

\begin{flushleft}
\mapleinline{inert}{2d}{Im(b(omega));}{%
$\Im (\mathrm{b}(\omega ))$%
} - imaginary part of b(
\mapleinline{inert}{2d}{omega;}{%
$\omega $%
})
To omit the singularity - the integral values is calculated using the
Cauchy principal value.
\end{flushleft}

\end{maplegroup}
\begin{maplegroup}
\begin{flushleft}
What's more - the theorem states that (2.2.1), (2.2.2) and (2.2.3a-b)
are mathematically equivalent. Proof with an exhausting review of both
the Fourier and Hilbert transforms has been described by Edward
Charles Titchmarsh in [15] - the theory is described through all book
chapters, but the theorem and its proof has been stated in chapter 5
about the conjugated integrals also called the Hilbert transforms.
\end{flushleft}

\end{maplegroup}
\begin{maplegroup}
\begin{flushleft}
We have nearly state the Kramers-Kronig relations. If we take a closer
look into equation (2.1.9) we will see that:
\end{flushleft}

\emptyline
\end{maplegroup}
\begin{maplegroup}
\begin{flushleft}
(2.2.4a) 
\mapleinline{inert}{2d}{chi[1,i,j](k,-omega);}{%
${\chi _{1, \,i, \,j}}(k, \, - \omega )$%
} = F[
\mapleinline{inert}{2d}{G[1,i,j];}{%
${G_{1, \,i, \,j}}$%
}(\textbf{r} ,-t)] = 
\mapleinline{inert}{2d}{int(int(G[1,i,j](ksi,tau)*exp(i*omega*tau),tau = 0 ..
infinity)*exp(-i*k*xi),xi = 0 .. infinity);}{%
$\int _{0}^{\infty }\int _{0}^{\infty }{G_{1, \,i, \,j}}(\mathit{
ksi}, \,\tau )\,e^{(i\,\omega \,\tau )}\,d\tau \,e^{( - i\,k\,\xi
 )}\,d\xi $%
} = [
\mapleinline{inert}{2d}{chi[1,i,j](k,omega);}{%
${\chi _{1, \,i, \,j}}(k, \,\omega )$%
}]{\large *}
\end{flushleft}

\end{maplegroup}
\begin{maplegroup}
\begin{flushleft}
(2.2.4b) 
\mapleinline{inert}{2d}{chi[1,i,j](k,-omega) = [chi[1,i,j](k,omega)];}{%
${\chi _{1, \,i, \,j}}(k, \, - \omega )=[{\chi _{1, \,i, \,j}}(k
, \,\omega )]$%
}{\large *}
\end{flushleft}

\end{maplegroup}
\begin{maplegroup}
\begin{flushleft}
where:
* - the symbol denoting the complex conjugate.
\end{flushleft}

\end{maplegroup}
\begin{maplegroup}
\begin{flushleft}
With such assumption we deduce that:
\end{flushleft}

\end{maplegroup}
\begin{maplegroup}
\begin{flushleft}
(2.2.5a) Re \{ 
\mapleinline{inert}{2d}{chi[1,i,j](k,omega);}{%
${\chi _{1, \,i, \,j}}(k, \,\omega )$%
} \} = Re \{
\mapleinline{inert}{2d}{chi[1,i,j](k,-omega);}{%
${\chi _{1, \,i, \,j}}(k, \, - \omega )$%
}\}
\end{flushleft}

\end{maplegroup}
\begin{maplegroup}
\begin{flushleft}
(2.2.5b) Im \{ 
\mapleinline{inert}{2d}{chi[1,i,j](k,omega);}{%
${\chi _{1, \,i, \,j}}(k, \,\omega )$%
} \} = -Im \{
\mapleinline{inert}{2d}{chi[1,i,j](-k,omega);}{%
${\chi _{1, \,i, \,j}}( - k, \,\omega )$%
}\}
\end{flushleft}

\end{maplegroup}
\begin{maplegroup}
\begin{flushleft}
Now the previous equation (2.2.3.a-b) can be written in a new form:
\end{flushleft}

\end{maplegroup}
\begin{maplegroup}
\begin{flushleft}
(2.2.6a) 
\mapleinline{inert}{2d}{Re(chi[1,i,j](k,omega)) =
2/pi*int(Omega*Im(chi[1,i,j](k,Omega))/(Omega^2-omega^2),Omega = 0 ..
infinity);}{%
$\Re ({\chi _{1, \,i, \,j}}(k, \,\omega ))=\frac {2\,\int _{0}^{
\infty }\frac {\Omega \,\Im ({\chi _{1, \,i, \,j}}(k, \,\Omega ))
}{\Omega ^{2} - \omega ^{2}}\,d\Omega }{\pi }$%
}
(2.2.6b) 
\mapleinline{inert}{2d}{Im(chi[1,i,j](k,omega)) =
(-2*omega/(-pi))*int(Re(chi[1,i,j](k,Omega))/(Omega^2-omega^2),Omega =
0 .. infinity);}{%
$\Im ({\chi _{1, \,i, \,j}}(k, \,\omega ))=( - \frac {2\,\omega 
}{ - \pi })\,\int _{0}^{\infty }\frac {\Re ({\chi _{1, \,i, \,j}}
(k, \,\Omega ))}{\Omega ^{2} - \omega ^{2}}\,d\Omega $%
}
\end{flushleft}

\end{maplegroup}
\begin{maplegroup}
\begin{flushleft}
For the first look the relations (2.2.6a-b) may look not so
impressive, but we must have in our mind that the real and imaginary
part of susceptibility describes two completely different optical
phenomena - the refraction of light and the absorption of light. If we
like to measure them - in most cases - we will need to perform two
different experiments. Kramers-Kronig relations allows us to imply -
that for now we only need to properly measure one of this two values -
and the other can be calculated on the basis of its results. This is
of course the half-true. Taking into consideration many in
uncertainities and experimental errors it is also an good idea to
measure both values - and then check if the theoretical model in which
we describe the optical properties of investigated material - is valid
and therefore we can proof its self-consistency.
\end{flushleft}

\end{maplegroup}
\begin{maplegroup}
\begin{flushleft}
\textit{Derivation of the Kramers-Kronig relation for linear optics.}
\end{flushleft}

\end{maplegroup}

\subsubsection{2.3 Simple nonlinear model}

\begin{maplegroup}
\begin{flushleft}
It is true that our knowledge about the linear interaction between the
light and matter has been investigated in recent several decades quite
well. We shall even say that this scientific area has been described
almost completely. The really interesting and important from the
scientific point of view is the area of nonlinear optics - when more
than a one light wave interact with matter in a definite, extremely
short period of time with a length of femto- or even attoseconds. When
two or more photons 'shoot' an atom or a molecule - we may observe
many new kinds of physical phenomena. From the quantum physics point
of view - each photon transports some amount of energy depending on
its frequency. More energy in a short time period - means a higher
chance for the molecule to modify its spatial structure, electron
energy or some other property. 
\end{flushleft}

\emptyline
\end{maplegroup}

\subsubsection{Pump-and-probe process}

\begin{maplegroup}
\begin{flushleft}
In a typical pump-and-probe process we observe two laser beams. One of
them is a strong signal - called the 'pump' - the second signal - the
'probe' - is much less intensive. We try to synchronize this two laser
beams in a following way - firstly a pump signal hits the sample with
its strong intensity - causing modifications in the sample properties.
Shortly after that - within a time period 
\mapleinline{inert}{2d}{Delta;}{%
$\Delta $%
}t - a low-intesive probe signal hits the modified sample and runs
through the detector. In a typical experiments the 
\mapleinline{inert}{2d}{Delta;}{%
$\Delta $%
}t time period should be adjustable. There usually are much more
detectors, but we are mostly interested in the polarization of the
probe signal. An important assumption for both pump and probe beams is
for them to be in the same polarization. The schematic diagram of a
pump-and-probe process has been described on the Figure 2.3.1. 
\end{flushleft}

\end{maplegroup}
\begin{maplegroup}
\begin{flushleft}

{\small 
[Figure 2.3.1] - The pump-and-probe process}
\end{flushleft}

\end{maplegroup}
\begin{maplegroup}
\begin{flushleft}
The key assumption in the pump-and-probe process is that separated and
independent probe signal does not influence the sample.
\end{flushleft}

\end{maplegroup}
\begin{maplegroup}
\emptyline
\end{maplegroup}

\subsubsection{Frequency Mixing Process:}

\begin{maplegroup}
\emptyline
\end{maplegroup}
\begin{maplegroup}
\emptyline
\end{maplegroup}
\begin{maplegroup}
\begin{flushleft}
For more ifno - please ask Profesor Samoc :)
\end{flushleft}

\end{maplegroup}
\begin{maplegroup}
\begin{flushleft}
\textit{Description of simple modification in linear model to obtain
the nonlinear model.}
\end{flushleft}

\end{maplegroup}

\subsubsection{2.4 Derivation of the Kramers-Kronig  relations for
simple nonlinear model}

\begin{maplegroup}
\begin{flushleft}
First we start from the upper Volterra-series expantion. But the we
jump to [16] and see, that there must be something more, that just a
simple K-K relations from the Titchmarsh theorem.
\end{flushleft}

\end{maplegroup}
\begin{maplegroup}
\begin{flushleft}
\textit{Mathematical derivation of the Kramers-Kronig relations for
simple nonlinear model.}
\end{flushleft}

\end{maplegroup}

\subsubsection{2.5 Overview of the simple quantum-perturbative model}

\begin{maplegroup}
\begin{flushleft}
What the heck is the perturbative approach in nonlinear optics? Here
we would like to describe the models derived by the quantum physics.
Not going into details, but to signalize how they mathematically look
like.
\end{flushleft}

\end{maplegroup}
\begin{maplegroup}
\begin{flushleft}
The theory of nonlinear optics is based mostly on a perturbative
approach introduced by Bloembergen [17], which is essentially a
semiclassical analogue of the Feynman graphs formalism [16] and which
generally adopts dipolar approximation. In the perturbative approach,
the nonlinear optical properties of a material are fully described by
nonlinear susceptibilities, which are obtained
by applying the Fourier transform to nonlinear Green functions, which
describe the higher order dynamics of the system. The nonlinear Green
function can be obtained using a generalization of Kubo�s linear
response theory. Results obtained using the perturbative theory are
generally in good agreement with the experimental data unless we
consider the interaction of matter with ultra-intense lasers, such as
those responsible for very high order harmonic-generation. A detailed
presentation of the perturbative theory for nonlinear optics and of
comparison with experimental data can be found in recent books by
Butcher and Cotter [18] and Boyd [19].
[126]. 
16. R. P. Feynman, �Space-time approach to quantum electrodynamics,�
Phys. Rev. 76, 769�789 (1949).
17. N. Bloembergen, Nonlinear Optics (Benjamin, New York, 1965).
18. P. N. Butcher and D. Cotter, Elements of Nonlinear Optics
(Cambridge University Press, Cambridge, 1990).
19. R. W. Boyd, Nonlinear Optics (Academic Press, New York, 1992).
126. R. Kubo, �Statistical-mechanical theory of irreversible
processes. I. General theory and simple applications to magnetic and
conduction problems,� J. Phys. Soc. Jpn. 12, 570�586 (1957).
\end{flushleft}

\end{maplegroup}
\begin{maplegroup}
\emptyline
\end{maplegroup}
\begin{maplegroup}
\begin{flushleft}
\textit{Brief overview of the perturbative model}
\end{flushleft}

\end{maplegroup}

\subsubsection{2.6 Derivation of the Kramers-Kronig  relations for
simple quantum-perturbative model}

\begin{maplegroup}
\begin{flushleft}
Here  
\end{flushleft}

\end{maplegroup}
\begin{maplegroup}
\begin{flushleft}
\textit{Mathematical derivation of the Kramers-Kronig relations for
simple perturbative model.}
\end{flushleft}

\end{maplegroup}

\subsubsection{2.7 Overview of the other models}

\begin{maplegroup}
\begin{flushleft}
Here we would like to state  - only from the physical point of view -
without formulating advanced mathematical models - that the theory of
K-K relations - or to say it precisely - the conjugated pairs of
Hilbert transforms - need to be structured and defined generaly for
both the linear and nonlinear case - there must be on valid theory -
it is not a good situation - when we start from one theory and then
"modify" it for some extra case - because it leads to misunderstanding
in general. As the title of this thesis contains the word -
understanding - we should at least try to express a new, general
theory of optical phenomena - because it is "only and even" - the
interaction of a chemical molecules with incoming photons in short or
long period of time.
\end{flushleft}

\end{maplegroup}
\begin{maplegroup}
\begin{flushleft}
\textit{Brief overview of the other model - 1. This subchapter may be
divided into several chapters.}
\end{flushleft}

\end{maplegroup}

\subsubsection{2.8 Derivation of the Kramers-Kronig  relations for
other models}

\begin{maplegroup}
\begin{flushleft}
This chapter is only a proposal - that we should avoid the parity
principle derived from the Titchmarsh theorem (that real/imag part
must be even/odd) and we will come back to the hilbert transform - the
integral from -infinity to +infinity. We also define the proposition
of K-K relations for the limited integration range from [17]:
\end{flushleft}

\end{maplegroup}
\begin{maplegroup}
\begin{flushleft}
\mapleinline{inert}{2d}{Im(f(omega)) = (-2*omega/Pi)*int(Re(f(Omega))/(Omega^2-omega^2),Omega
= omega[a] .. omega[b]);}{%
$\Im (\mathrm{f}(\omega ))=( - \frac {2\,\omega }{\pi })\,\int _{
{\omega _{a}}}^{{\omega _{b}}}\frac {\Re (\mathrm{f}(\Omega ))}{
\Omega ^{2} - \omega ^{2}}\,d\Omega $%
} 
\end{flushleft}

\end{maplegroup}
\begin{maplegroup}
\begin{flushleft}
\textit{Mathematical derivation of the Kramers-Kronig relations for
other models. This subchapter may be divided into several chapters.}
\end{flushleft}

\end{maplegroup}

\subsection{3. ZScan Measurments (understanding)}

\subsubsection{3.1 Overview of the z-scan technique}

\begin{maplegroup}
\begin{flushleft}
In Nonlinear optics a z-scan measurement is used to measure the
non-linear index 
\mapleinline{inert}{2d}{n[2];}{%
${n_{2}}$%
} Kerr nonlinearity and the non-linear absorption coefficient 
\mapleinline{inert}{2d}{Delta(a)*a;}{%
$\Delta (a)\,a$%
} via the "open" and "closed" methods respectively. 
\end{flushleft}

\begin{flushleft}
More text here... 
\end{flushleft}

\end{maplegroup}
\begin{maplegroup}
\begin{flushleft}
\textit{Description of the z-scan technique, description of the z-scan
experiment built in Wroclaw, since summer 2010.}
\end{flushleft}

\end{maplegroup}

\subsubsection{3.2 Overview of the derivation }

\begin{maplegroup}
\begin{flushleft}
Text here... 
\end{flushleft}

\end{maplegroup}
\begin{maplegroup}
\begin{flushleft}
\textit{Description how the gathered data is transformed into final
chart. Some mathematical derivations.}
\end{flushleft}

\end{maplegroup}

\subsubsection{3.3 Defition of the main problems}

\begin{maplegroup}
\begin{flushleft}
What effects we measure within the z-scan technique in general? 
Howto model them, howto gather experimental data from the 
\end{flushleft}

\end{maplegroup}
\begin{maplegroup}
\begin{flushleft}
\textit{Discussion about the z-scan technique, its advantages and
disadvantages for determination both real and imaginary part of }
\end{flushleft}

\end{maplegroup}

\subsection{4. Simpson and trapezoidal quadrature combined with cubic
interpolation - HTRAN (solving)}

\subsubsection{Overview of the HTRAN}

\begin{maplegroup}
\begin{flushleft}
We shall start from the simplest possible numerical calculation of
integral - so the simpson and trapezoidal quadrature. The algorithm is
based on report by I. J. Weinberg [18] written for CDC 6600
supercomputer.
\end{flushleft}

\end{maplegroup}
\begin{maplegroup}
\begin{flushleft}
\textit{Overview of the Cauchy Principal Value method.}
\end{flushleft}

\end{maplegroup}

\subsubsection{HTRAN for simple linear model - results}

\begin{maplegroup}
\begin{flushleft}
Text and graphs here.... 
\end{flushleft}

\end{maplegroup}
\begin{maplegroup}
\begin{flushleft}
\textit{Example cases for solving the Kramers-Kronig relations in
linear model using HTRAN - with short conclusions.}
\end{flushleft}

\end{maplegroup}

\subsubsection{HTRAN for simple nonlinear model - results}

\begin{maplegroup}
\begin{flushleft}
Text and graphs here... 
\end{flushleft}

\end{maplegroup}
\begin{maplegroup}
\begin{flushleft}
Example cases for solving the Kramers-Kronig relations in simple
nonlinear model using HTRAN - with short conslusions.
\end{flushleft}

\end{maplegroup}

\subsubsection{HTRAN for simple quantum-perturbative model - results}

\begin{maplegroup}
\begin{flushleft}
Text and graphs here... 
\end{flushleft}

\end{maplegroup}
\begin{maplegroup}
\begin{flushleft}
Example cases for solving the Kramers-Kronig relations in simple
\textit{quantum-}perturbative model using HTRAN - with short
conslusions.
\end{flushleft}

\end{maplegroup}

\subsubsection{HTRAN for other models - results}

\begin{maplegroup}
\begin{flushleft}
Text and graphs here... 
\end{flushleft}

\end{maplegroup}
\begin{maplegroup}
\begin{flushleft}
Example cases for solving the Kramers-Kronig relations for other
models using HTRAN - with short conslusions.
\end{flushleft}

\end{maplegroup}

\subsection{5. Newton-Cotes quadrature (solving)}

\subsubsection{Overview of the NC quadrature of arbitrary degree}

\begin{maplegroup}
\begin{flushleft}
Newton-Cotes formulae for numerical integration is taken into
consideration. We compare formulae for an arbitrary degree - having in
our mind, that the choice between formulas of high and low degree must
be undertaken with the awareness of numerical errors that may arise
\end{flushleft}

\end{maplegroup}
\begin{maplegroup}
\begin{flushleft}
\textit{The algorithm is ready, only its description must be expanded}
\end{flushleft}

\end{maplegroup}

\subsubsection{NC for simple linear model - results}

\begin{maplegroup}
\begin{flushleft}
Text and graphs here.... 
\end{flushleft}

\end{maplegroup}
\begin{maplegroup}
\begin{flushleft}
\textit{Example cases for solving the Kramers-Kronig relations in
linear model using HTRAN - with short conclusions.}
\end{flushleft}

\end{maplegroup}

\subsubsection{NC for simple nonlinear model - results}

\begin{maplegroup}
\begin{flushleft}
Text and graphs here... 
\end{flushleft}

\end{maplegroup}
\begin{maplegroup}
\begin{flushleft}
Example cases for solving the Kramers-Kronig relations in simple
nonlinear model using NC - with short conslusions.
\end{flushleft}

\end{maplegroup}

\subsubsection{NC for simple quantum-perturbative model - results}

\begin{maplegroup}
\begin{flushleft}
Text and graphs here... 
\end{flushleft}

\end{maplegroup}
\begin{maplegroup}
\begin{flushleft}
Example cases for solving the Kramers-Kronig relations in simple
\textit{quantum-}perturbative model using NC - with short conslusions.
\end{flushleft}

\end{maplegroup}

\subsubsection{NC for other models - results}

\begin{maplegroup}
\begin{flushleft}
Text and graphs here... 
\end{flushleft}

\end{maplegroup}
\begin{maplegroup}
\begin{flushleft}
Example cases for solving the Kramers-Kronig relations for other
models using NC - with short conslusions.
\end{flushleft}

\end{maplegroup}

\subsection{6. Clenshaw-Curtis quadrature - discussion (solving)}

\subsubsection{Overview of the Clenshaw-Curtis iterations}

\begin{maplegroup}
\begin{flushleft}
Text here... 
\end{flushleft}

\end{maplegroup}
\begin{maplegroup}
\begin{flushleft}
\textit{Overview of the }Clenshaw-Curtis iterations\textit{ method.}
\end{flushleft}

\end{maplegroup}

\subsubsection{CCI for simple linear model}

\begin{maplegroup}
\begin{flushleft}
Text and graphs here.... 
\end{flushleft}

\end{maplegroup}
\begin{maplegroup}
\begin{flushleft}
\textit{Example cases for solving the Kramers-Kronig relations in
linear model using }CCI\textit{ - with short conclusions.}
\end{flushleft}

\end{maplegroup}

\subsubsection{CCI for simple nonlinear model}

\begin{maplegroup}
\begin{flushleft}
Text and graphs here.... 
\end{flushleft}

\end{maplegroup}
\begin{maplegroup}
\begin{flushleft}
\textit{Example cases for solving the Kramers-Kronig relations in
simple nonlinear model using CCI - with short conslusions.}
\end{flushleft}

\end{maplegroup}

\subsubsection{CCI for simple quantum-perturbative model}

\begin{maplegroup}
\begin{flushleft}
Text and graphs here.... 
\end{flushleft}

\end{maplegroup}
\begin{maplegroup}
\begin{flushleft}
\textit{Example cases for solving the Kramers-Kronig relations in
simple quantum-perturbative model using CCI - with short conslusions.}
\end{flushleft}

\end{maplegroup}

\subsubsection{CCI for other models}

\begin{maplegroup}
\begin{flushleft}
Text and graphs here.... 
\end{flushleft}

\end{maplegroup}
\begin{maplegroup}
\begin{flushleft}
\textit{Example cases for solving the Kramers-Kronig relations for
other models using CCI - with short conslusions.}
\end{flushleft}

\end{maplegroup}

\subsection{7. Fast Hartley transforms (solving)}

\subsubsection{Overview of the FTH}

\begin{maplegroup}
\begin{flushleft}
This method is faster than the fast Fourier transform because it works
also in O(n log n) time, but it works only in the real domain
\end{flushleft}

\end{maplegroup}
\begin{maplegroup}
\begin{flushleft}
\textit{Overview of the fast Hartley transorms, the algorithm in
O(n\symbol{94}2) is ready, we need to enhance it to O(n log n)}
\end{flushleft}

\end{maplegroup}

\subsubsection{FTH for simple linear model}

\begin{maplegroup}
\begin{flushleft}
Text and graphs here.... 
\end{flushleft}

\end{maplegroup}
\begin{maplegroup}
\begin{flushleft}
\textit{Example cases for solving the Kramers-Kronig relations in
linear model using  - FTH with short conclusions.}
\end{flushleft}

\end{maplegroup}

\subsubsection{FTH for simple nonlinear model}

\begin{maplegroup}
\begin{flushleft}
Text and graphs here.... 
\end{flushleft}

\end{maplegroup}
\begin{maplegroup}
\begin{flushleft}
\textit{Example cases for solving the Kramers-Kronig relations in
simple nonlinear model using FTH - with short conslusions.}
\end{flushleft}

\end{maplegroup}

\subsubsection{FTH for simple quantum-perturbative model}

\begin{maplegroup}
\begin{flushleft}
Text and graphs here.... 
\end{flushleft}

\end{maplegroup}
\begin{maplegroup}
\begin{flushleft}
\textit{Example cases for solving the Kramers-Kronig relations in
simple quantum-perturbative model using FTH - with short conslusions.}
\end{flushleft}

\end{maplegroup}

\subsubsection{FTH for other models}

\begin{maplegroup}
\begin{flushleft}
Text and graphs here.... 
\end{flushleft}

\end{maplegroup}
\begin{maplegroup}
\begin{flushleft}
Example cases for solving the Kramers-Kronig relations for other
models using FTH - with short conslusions.
\end{flushleft}

\end{maplegroup}

\subsection{8. Hermite-Hilbert transform (solving)}

\subsubsection{Overview of the HHT}

\begin{maplegroup}
\begin{flushleft}
Hermite-Hilbert transform approach base on the precalcution of
already-known Hermite polynomials and Hermite base of orthogonal
functions
\end{flushleft}

\end{maplegroup}
\begin{maplegroup}
\begin{flushleft}
\textit{Overview of the Hermite-Hilbert transorms and overview of the
precalculation of the Hermite polynomials in a possibly short
time/memory.}
\end{flushleft}

\end{maplegroup}

\subsubsection{HHT for simple linear model}

\begin{maplegroup}
\begin{flushleft}
Text and graphs here.... 
\end{flushleft}

\end{maplegroup}
\begin{maplegroup}
\begin{flushleft}
\textit{Example cases for solving the Kramers-Kronig relations in
linear model using  - HHT with short conclusions.}
\end{flushleft}

\end{maplegroup}

\subsubsection{HHT for simple nonlinear model}

\begin{maplegroup}
\begin{flushleft}
Text and graphs here.... 
\end{flushleft}

\end{maplegroup}
\begin{maplegroup}
\begin{flushleft}
\textit{Example cases for solving the Kramers-Kronig relations in
simple nonlinear model using HHT - with short conslusions.}
\end{flushleft}

\end{maplegroup}

\subsubsection{HHT for simple quantum-perturbative model}

\begin{maplegroup}
\begin{flushleft}
Text and graphs here.... 
\end{flushleft}

\end{maplegroup}
\begin{maplegroup}
\begin{flushleft}
\textit{Example cases for solving the Kramers-Kronig relations in
simple quantum-perturbative model using HHT - with short conslusions.}
\end{flushleft}

\end{maplegroup}

\subsubsection{HHT for other models}

\begin{maplegroup}
\begin{flushleft}
Text and graphs here.... 
\end{flushleft}

\end{maplegroup}
\begin{maplegroup}
\begin{flushleft}
Example cases for solving the Kramers-Kronig relations for other
models using \textit{HHT} - with short conslusions.
\end{flushleft}

\end{maplegroup}

\subsection{9. Fourier-series (solving)}

\subsubsection{Overview of the Fouries-series baser method}

\begin{maplegroup}
\begin{flushleft}
Fourier series are similar to the Hermite-Hilbert transform. We derive
their mathematical and numerical formulation
\end{flushleft}

\end{maplegroup}
\begin{maplegroup}
\begin{flushleft}
\textit{The algorithm is ready, results will be generated - results
are quite well for finite signals/spectras}
\end{flushleft}

\end{maplegroup}

\subsubsection{Fouries-series for simple linear model}

\begin{maplegroup}
\begin{flushleft}
Text and graphs here.... 
\end{flushleft}

\end{maplegroup}
\begin{maplegroup}
\begin{flushleft}
\textit{Example cases for solving the Kramers-Kronig relations in
linear model using Fourier-series - with short conclusions.}
\end{flushleft}

\end{maplegroup}

\subsubsection{Fouries-series for simple nonlinear model}

\begin{maplegroup}
\begin{flushleft}
Text and graphs here.... 
\end{flushleft}

\end{maplegroup}
\begin{maplegroup}
\begin{flushleft}
\textit{Example cases for solving the Kramers-Kronig relations in
simple nonlinear model using Fourier-series - with short conslusions.}
\end{flushleft}

\end{maplegroup}

\subsubsection{Fouries-series for simple quantum-perturbative model}

\begin{maplegroup}
\begin{flushleft}
Text and graphs here.... 
\end{flushleft}

\end{maplegroup}
\begin{maplegroup}
\begin{flushleft}
\textit{Example cases for solving the Kramers-Kronig relations in
simple quantum-perturbative model using Fourier-series - with short
conslusions.}
\end{flushleft}

\end{maplegroup}

\subsubsection{Fouries-series for other models}

\begin{maplegroup}
\begin{flushleft}
Text and graphs here.... 
\end{flushleft}

\end{maplegroup}
\begin{maplegroup}
\begin{flushleft}
Example cases for solving the Kramers-Kronig relations for other
models using \textit{Fourier-series} - with short conslusions.
\end{flushleft}

\end{maplegroup}

\subsection{10. MATLAB � out-of-the-box functions (solving)}

\subsubsection{Overview of the MATLAB � interior functions}

\begin{maplegroup}
\begin{flushleft}
Why not take into consideration the already built-in numerical methods
from MATLAB. We compare the results obtained with:
- quadgk() - Gauss-Kronrod Quadrature;
- hilbert() - fast Hilbert transform based on both FFT and IFFT
\end{flushleft}

\end{maplegroup}
\begin{maplegroup}
\begin{flushleft}
\textit{Overview of the method using }MATLAB � interior functions.
\end{flushleft}

\end{maplegroup}

\subsubsection{MIF for simple linear model}

\begin{maplegroup}
\begin{flushleft}
Text and graphs here.... 
\end{flushleft}

\end{maplegroup}
\begin{maplegroup}
\begin{flushleft}
\textit{Example cases for solving the Kramers-Kronig relations in
linear model using MIF - with short conclusions.}
\end{flushleft}

\end{maplegroup}

\subsubsection{MIF for simple nonlinear model}

\begin{maplegroup}
\begin{flushleft}
Text and graphs here.... 
\end{flushleft}

\end{maplegroup}
\begin{maplegroup}
\begin{flushleft}
Example cases for solving the Kramers-Kronig relations in simple
nonlinear model using \textit{MIF} - with short conslusions.
\end{flushleft}

\end{maplegroup}

\subsubsection{MIF for simple quantum-perturbative model}

\begin{maplegroup}
\begin{flushleft}
Text and graphs here.... 
\end{flushleft}

\end{maplegroup}
\begin{maplegroup}
\begin{flushleft}
Example cases for solving the Kramers-Kronig relations in simple
quantum-perturbative model using \textit{MIF} - with short
conslusions.
\end{flushleft}

\end{maplegroup}

\subsubsection{MIF for other models}

\begin{maplegroup}
\begin{flushleft}
Text and graphs here.... 
\end{flushleft}

\end{maplegroup}
\begin{maplegroup}
\begin{flushleft}
Example cases for solving the Kramers-Kronig relations for other
models using \textit{MIF} - with short conslusions.
\end{flushleft}

\end{maplegroup}

\subsection{11. General comparison of used numerical methods
(understanding)}

\begin{maplegroup}
\begin{flushleft}
\textit{Nice charts of numerical comparisons here :)}
\end{flushleft}

\end{maplegroup}

\subsection{12. Conclusion (understanding)}

\subsubsection{Model conclusion}

\begin{maplegroup}
\begin{flushleft}
We have presented both the linear and nonlinear model with comments
that arise from the literature. The main question remains - what is
the true and valid model for describing the interation of finite
amount of photons with chemical molecules
\end{flushleft}

\end{maplegroup}
\begin{maplegroup}
\begin{flushleft}
\textit{Final discussion about the used models}
\end{flushleft}

\end{maplegroup}

\subsubsection{Numerical conclusion}

\begin{maplegroup}
\begin{flushleft}
We have presented several methods and the comparison of their
accuracy, convergency and speed. Which method is the best and why?
\end{flushleft}

\end{maplegroup}
\begin{maplegroup}
\begin{flushleft}
\textit{Final discussion about the used numerical methods and their
accuracy.}
\end{flushleft}

\end{maplegroup}

\subsubsection{Z-scan technique conclusion}

\begin{maplegroup}
\begin{flushleft}
The Z-scan experiment was the most important motivation for this
thesis to be written. It is of course one of many experimental setups
in nonlinear optics - but the questions are:
- how both the experimental and numerical accuracy influence on the
final results
- how to avoid this problems in the z-scan experiment?
\end{flushleft}

\end{maplegroup}
\begin{maplegroup}
\begin{flushleft}
\textit{Final discussion for the building better z-scan experiment
(Z-scan 2.0 ? :))}
\end{flushleft}

\end{maplegroup}

\subsection{13. Acknowledgements}

\begin{maplegroup}
\begin{flushleft}
Text here... 
\end{flushleft}

\end{maplegroup}
\begin{maplegroup}
\begin{flushleft}
\textit{Final acknowledgements}
\end{flushleft}

\end{maplegroup}

\subsection{14. References}

\begin{maplegroup}
\begin{flushleft}
[1] G. Tsigaridas, M. Fakis, I. Polyzos, P. Persephonis and V.
Giannetas, \textit{"Z-scan technique through beam radius
measurements"}, Applied Physics B: Lasers and Optics
Volume 76, Number 1, 83-86, DOI: 10.1007/s00340-002-1067-5
\end{flushleft}

\end{maplegroup}
\begin{maplegroup}
\begin{flushleft}
[2] Robert W. Boyd, \textit{"Nonlinear Optics, Third Edition"},
Academic Press, Elsevier, March 2008, ISBN: 978-0-12-369470-6
\end{flushleft}

\end{maplegroup}
\begin{maplegroup}
\begin{flushleft}
[3] Marek Samoc, \textit{"Third-order nonlinear optical materials:
practical issues and theoretical challenges"}, Journal of Molecular
Modeling, DOI: 10.1007/s00894-010-0856-8
\end{flushleft}

\end{maplegroup}
\begin{maplegroup}
\begin{flushleft}
[4] Marek Samoc, A. Samoc, B. Luther-Davies, M. G. Humphrey, M. P.
Cifuentes, \textit{"Nonlinear absorption: materials and mechanisms",
}Proceedings of the Symposium on Photonics Technologies for 7th
Framework Program Wroclaw\textit{ (2006),} Issue: October, Pages:
12�14,  ISBN: 83-7085-970-4 
[5] Robert W Boyd, John E Sipe, Peter W Milonni, \textit{"Chirality
and polarization effects in nonlinear optics"}, 2004 J. Opt. A: Pure
Appl. Opt. 6 S14 DOI: 10.1088/1464-4258/6/3/002
[6] A. Nov�k - \textit{"Identification of Nonlinear Systems: Volterra
Series Simplification"}, Acta Polytechnica Vol. 47 No. 4�5/2007
[7] S. T. Chui and Hong-ru Ma - \textit{"Local-field correction for
nonlinear optical coefficients"}, Phys. Rev. B 47, 6293�6298 (1993),
DOI:10.1103/PhysRevB.47.6293
[8] Paras N. Prasad, David J. Williams - \textit{"Introduction to
Nonlinear Optical Effects in Molecules and Polymers"} ISBN:
978-0-471-51562-3
[9] M. D. Levenson* and N. Bloembergen  - \textit{"Dispersion of the
nonlinear optical susceptibility tensor in centrosymmetric media"},
Phys. Rev. B 10, 4447�4463 (1974) DOI:10.1103/PhysRevB.10.4447
[10] Frank Tr�ger, "Springer Handbook of Lasers and Optics", Springer
2007, ISBN: 0387955798, partially available on Google Books service:
http://www.google.com/books?id=YOHJGz-9UNoC
[11] Nicolaas Bloembergen (Nobel Prize for Physics (1981)) -
\textit{"Nonlinear optics - 4th Edition" }- Singapore, World
Scientific, 1996, ISBN: 981-02-2599-7 (pbk), partially available on
the Google Books service: http://www.google.com/books?id=VdwBT13l05EC
[12] Yitzhak Katznelson - \textit{"An introduction to harmonic
analysis - third edition" - }Cambridge Unviversity Press, 2004, ISBN:
0-521-54359-2 (pbk), partially available on the Google Book servise:
http://www.google.com/books?id=gkpUE\_m5vvsC
[13] Palffy-Muhoray, Course materials for LC Optics and Photonics,
CPHY 6/74495, Spring, 2011, website:
http://mpalffy.lci.kent.edu/Optics/
[14] Valerio Lucarini, Jarkko J. Saarinen, Kai-Erik Peiponen, and Erik
M. Vartiainen (2005), \textit{"Kramers-Kronig relations in Optical
Materials Research",} Heidelberg: Springer. ISBN 3-540-23673-2.
\end{flushleft}

\begin{flushleft}
[15] Edward Charles Titchmarsh, "Introduction to the theory of Fourier
integrals - second edition", Oxford at the Clarendon Press, 1948
\end{flushleft}

\begin{flushleft}
[16] H. N. Yum, M. S. Shahriar, "\textit{Pump�probe model for the
Kramers�Kronig relations in a laser}", J. Opt. 12 (2010) 104018 (6pp),
DOI:10.1088/2040-8978/12/10/104018
\end{flushleft}

\begin{flushleft}
[17] H. Tuononen, E. Gornov, J. A. Zeitler, J. Aaltonen, and K.-E.
Peiponen, \textit{"Using modified Kramers-Kronig relations to test
transmission spectra of porous media in THz-TDS"}, Optics Letters,
Vol. 35, Issue 5, pp. 631-633 (2010), DOI:10.1364/OL.35.000631
\end{flushleft}

\begin{flushleft}
[18] I. J. Weinberg, "\textit{Hilbert Transform by Numerical
Integration}", ROME AIR DEVELOPMENT CENTER GRIFFISS AFB N Y, 1979,
Report Number:A480860
\end{flushleft}

\end{maplegroup}
\begin{maplegroup}
\begin{flushleft}
[17-100] Nice positions here...
\end{flushleft}

\end{maplegroup}
\emptyline
\end{document}
%% End of Maple 14.00 Output
