%% Krzysztof Parjaszewski - the summary of prepared Master Thesis
\documentclass[10pt,twoside,a4paper]{article}
\usepackage{fullpage}
\addtolength{\hoffset}{-1.0cm}
\addtolength{\voffset}{-1.0cm}
\addtolength{\textwidth}{2cm}
\pagestyle{empty}
\title{Numerical evaluation of the Hilbert transform used to better understand 
and solve the Kramers-Kronig relations in nonlinear optics - summary}
 \author{Krzysztof Parjaszewski}
 \date{}
\def\Xint#1{\mathchoice
{\XXint\displaystyle\textstyle{#1}}%
{\XXint\textstyle\scriptstyle{#1}}%
{\XXint\scriptstyle\scriptscriptstyle{#1}}%
{\XXint\scriptscriptstyle\scriptscriptstyle{#1}}%
\!\int}
\def\XXint#1#2#3{{\setbox0=\hbox{$#1{#2#3}{\int}$ }
\vcenter{\hbox{$#2#3$ }}\kern-.6\wd0}}
\def\ddashint{\Xint=}
\def\dashint{\Xint-}
\begin{document}
\maketitle
\paragraph{}
The motivation for this work comes from the real problem state by
physicists and chemists building the valid theoretical models to
describe the interaction between light and matter due to both
non-intensive and intensive light radiation. The starting point is the
fundamental physical law - the causality principle - which states that
the effect cannot precede the cause. This simple assumption leads us
to the theorem stated by Edward Charles Titchmarsh in 1948 about the
pairs of conjugated integrals - Hilbert transforms. Titchmarsh has
noted that for any casual and convergent response signal $\Phi (t)$ belonging to the Lebesgue 
$L^{2}$ space over real line using the Fourier transform we can calculate
the holomorphic $\chi (\omega )$ spectrum which belongs to the Hardy 
$H^{2}$ space over upper half-plane with the property that real and
imaginary part of $\chi (\omega )$ are conjugated - we can calculate one from each other using the
Hilbert transforms. This observation - together with the signal
response-theory allows us to investigate the properties of the medium
response to the periodic input signal - both in time and frequency
domain. The Kramers-Kronig relations for both real and imaginary part
of optical susceptibility $\chi (\omega )$  come from the simple assumption, that real part is even and
imaginary part is an odd function. We can therefore modify the Hilbert
transform equation to obtain the most popular form of the
Kramers-Kronig relations:
$$\Re \{ \chi (\omega ) \}=\frac {2}{\pi} \,\dashint _{ - \infty }^{\infty }
\frac {\omega^{\prime} \,\Im (\chi (\omega^{\prime} ))}{{\omega^{\prime}} ^{2} - \omega ^{2}}
\,d \omega^{\prime}  $$


$$\Im (\chi (\omega ))=\frac {2}{\pi}\,\omega \,\dashint _{ - \infty }^{
\infty }\frac {\Re (\chi (\omega^{\prime} ))}{{\omega^{\prime}} ^{2} - \omega ^{2}}\,
d\omega^{\prime}  $$

In this thesis we handle two problems. First concerns the proper
numerical evaluation of a singular and improper integral defined in
the Hilbert transform operator.The second problem concerns the
questions stated by physicists - how to properly use this mathematical
tools in a typical experiment and model construction in optical
research. We present the comparison of several implementations of
numerical evaluations of the Hilbert transform:
\begin{enumerate}
 \item Easy implementations - on the comparison needs:
  \begin{itemize}
   \item Newton-Cotes qudrature of an arbitrary degree - comparison,
   \item HTRAN procedure - numerical trapezoidal rule mixed with the
Simpson's rule and the cubic interpolation, based on [1].
  \end{itemize}
 \item Advanced implementations:
  \begin{itemize}
   \item Generalized Clenshaw-Curtis quadrature,
   \item Hilbert transform based on fast Hartley transforms,
   \item method based on approximation with the orthonormal Hermite
polynomials and Hermite functions,
   \item method based on approximation with the Fourier series.
  \end{itemize}
 \item Testing the out-of-the-box MATLAB-implemented routines:
  \begin{itemize}
   \item quadgk() - Gauss-Kronrod quadrature,
   \item hilbert() - fast Hilbert transfom based on fast Fourier transform
and inverse fast Fourier transform.
  \end{itemize}
\end{enumerate}

The given physical models for both linear and nonlinear optics are
analysized and validated. We formulate good practises for scientists
interested in subject of optical experiments. Finally we made the
conclusions about the stability, accuracy, advantages and
disadvantages of developed implementations in further research in
nonlinear optics.
\\ \\ 
$[1]$ I. J. Weinberg, \textit{Hilbert Transform by Numerical
Integration}", ROME AIR DEVELOPMENT CENTER GRIFFISS AFB N Y, 1979,
Report Number:A480860
\end{document}
