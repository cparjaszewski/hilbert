%% Krzysztof Parjaszewski - the summary of prepared Master Thesis
\documentclass{article}
\usepackage{fullpage}
\title{Numerical calculation of the Hilbert transform applied for
understanding and solving the Kramers-Kronig relations in nonlinear
optics - summary}
 \author{Krzysztof Parjaszewski}
 \date{}
\def\Xint#1{\mathchoice
{\XXint\displaystyle\textstyle{#1}}%
{\XXint\textstyle\scriptstyle{#1}}%
{\XXint\scriptstyle\scriptscriptstyle{#1}}%
{\XXint\scriptscriptstyle\scriptscriptstyle{#1}}%
\!\int}
\def\XXint#1#2#3{{\setbox0=\hbox{$#1{#2#3}{\int}$ }
\vcenter{\hbox{$#2#3$ }}\kern-.6\wd0}}
\def\ddashint{\Xint=}
\def\dashint{\Xint-}
\begin{document}
\maketitle
\paragraph{}
The motivation for this work comes from the real problem state by
physicists and chemists building the valid theoretical models to
describe the interaction between light and matter due to both
non-intensive and intensive light radiation. The starting point is the
fundamental physical law - the causality principle - which states that
the effect cannot precede the cause. This simple assumption leads us
to the Titchmarsh theorem about the Hilbert transform - and together with
an advanced signal response-theory we can investigate the properties of
the medium response to the periodic input signal - both in time and
frequency domain. Modifications of the Hilbert transform for the
optical susceptibility $\chi(\omega)$ are known as the Kramers-Kronig relations - which are defined for
both real and imaginary part of susceptibility in the frequency-domain:

$$\Re \{ \chi (\omega ) \}=\frac {2}{\pi} \,\dashint _{ - \infty }^{\infty }
\frac {\omega^{\prime} \,\Im (\chi (\omega^{\prime} ))}{{\omega^{\prime}} ^{2} - \omega ^{2}}
\,d \omega^{\prime}  $$


$$\Im (\chi (\omega ))=\frac {2}{\pi}\,\omega \,\dashint _{ - \infty }^{
\infty }\frac {\Re (\chi (\omega^{\prime} ))}{{\omega^{\prime}} ^{2} - \omega ^{2}}\,
d\omega^{\prime}  $$

The value is assigned to this integrals using the Cauchy principal value method. In this
thesis we handle two problems. First concirns numerical problems with
calculation of such singular and improper integral. The second problem
concirns the questions stated by physicists - how to properly use this
mathematical tools in a typical experiment (errors, few data) and to construct valid model in
optical research. We present the comparison of several implementations
of numerical calculations of the Hilbert transform:
\begin{itemize}
    \item Clenshaw-Curtis quadrature,
    \item Hilbert transform based on fast Hartley transforms,
    \item Newton-Cotes qudrature of an arbitrary degree,
    \item Numerical trapezoidal rule mixed with the Simpson's rule and the
cubic interpolation,
    \item method based on function approximation with the orthonormal Hermite
polynomials and Hermite functions,
	\item method based on approximation with the Fourier series.
\end{itemize}
We also test the out-of-the-box MATLAB-implemented routines:
\begin{itemize}
	\item quadgk(),
	\item hilbert() - based on FFT.
\end{itemize}
The given physical models for both linear and nonlinear optics are
analysized and validated. We formulate good practises for scientists
interested in subject of optical experiments. Finally we made the
conclusions about the numerical stability, advantages and
disadvantages of developed implementations in further research in
nonlinear optics.
\end{document}
